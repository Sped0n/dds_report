


\setcounter{page}{1}
\pagenumbering{arabic}
\songti\xiaosi{
\chapter{设计概述}
\begin{spacing}{1.5}

\section{设计目的}
	DDS 是直接数字频率合成器(Direct Digital Synthesizer)的英文缩写,是一项关键的数 字化技术。与传统频率合成器相比,DDS 具有低成本,低功耗,高分辨率和快速转换时间 的优点,广泛使用在电信与电子仪器领域,是实现设备全数字化的一个关键技术。

	本实验设计 DDS 发生器旨在掌握 Xilinx FPGA 开发流程,掌握 DDS 原理,并且获取 自己的 DDS 发生器便于后续课程学习中应用开发。

\section{应用领域}
DDS 的应用领域十分广泛,常用于信号处理,电路系统分析,仪器仪表设计等领域中, 在通信领域中应用尤其广泛。2020 年江苏省大学生 TI 杯电子设计竞赛中 E 题《放大器非线 性失真研究装置》的信号源就应用了 DDS 作为正弦信号发生器来研究放大电路的四种失真情况。

\section{设计要求和性能指标}
\begin{enumerate}
	\item 能够产生正弦波、方波、三角波、锯齿波四种波形并且在示波器上面观察,并且 可以人为控制波形的输出种类。
	\item 可以用开关作为频率控制字控制输出波形的频率。
	\item 可以基于 DDS 原理,计算输出波形频率的理论值并在数码管上面显示。
	\item 可以对输出的波形进行测频,将测出的实际频率在 7 段数码管上面显示,二者之间误差不得超过 5Hz。
\end{enumerate}

\section{设计优点}
	与传统 DDS 芯片相比,传统 DDS 芯片只能产生方波、正弦波,而利用 FPGA 内部资源实 现 DDS 因其设计灵活,波形不受限制而逐渐成为热门。
\end{spacing}}



