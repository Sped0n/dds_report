\songti\xiaosi{
\chapter{各子模块功能的编写}
\begin{spacing}{1.5}
    \section{DDS子模块的编写}
        \subsection{分频模块的编写}
            分频模块的作用为把系统时钟按照一定的比例进行缩小,在数字逻辑电路中主要实现 的元件为 D 触发器,在实际设计中常用 PLL 锁相环 IP 核来实现。本次设计中,根据课程要 求,我们只需要获得 10KHz 、 0.5Hz 、50Hz 和 100Hz 的时钟,若使用 PLL 锁相环未免有些大材小用,同时 还会给系统增加负担,减慢综合器运行速度,因此我采用直接书写模块的方式进行实现。编 程的思路大致如下:

            首先计算出分频后要求的时钟的周期,然后将这一周期与系统时钟的周期对比,计算 系统时钟到达多少次上升沿或下降沿可以达到分频时钟要求的周期。比如,想要得到 10kHz 的时钟,那么我们就要计算在 100MHz 时钟的作用下有多少个上升或下降沿通过,计算后, 是 10000 个,然后需要将 10000 除以 2 再减去 1 作为变化的指标,当 100MHz 产生的时钟 上升沿数量到达这个数值时,输出时钟就会在原来初始值的基础上翻转一次,从而通过计数 的方式实现分频功能,这一方法相比于 PLL 锁相环可以加快综合速度,相比于 D 触发器级 联可以减少复杂度,因此较为合理。0.5Hz 时钟产生的方法同理。因此我们可以使用上面的 方法来编写分频程序,具体程序如下:
			\begin{lstlisting}[language=Verilog]
            module divclk(
                input clk,
                output reg out_clk1,
                output reg out_clk2,
                output reg out_clk3,
                output reg out_clk4
                );
            
                reg[31:0] out_clk1_cnt = 0;
                reg[31:0] out_clk2_cnt = 0;
                reg[31:0] out_clk3_cnt = 0;
                reg[31:0] out_clk4_cnt = 0;

                initial
			\end{lstlisting}
			\begin{lstlisting}[language=Verilog]
                begin
                    out_clk1 = 0;
                    out_clk2 = 0;
                    out_clk3 = 0;
                    out_clk4 = 0;
                end
            
                parameter div_1ms = 49999;
                parameter div_20ms = 999999;
                parameter div_10khz = 4999;
                parameter div_05hz = 99999999;
                // 1ms clock
                always@(posedge clk)
                begin
                    out_clk1_cnt <= out_clk1_cnt + 1;
                    if (out_clk1_cnt == div_1ms)
                    begin
                        out_clk1 <= ~out_clk1;
                        out_clk1_cnt <= 0;
                    end
                end
                // 20ms clock
                always@(posedge clk)
                begin
                    out_clk2_cnt <= out_clk2_cnt + 1;
                    if (out_clk2_cnt == div_20ms) 
                    begin
                    out_clk2 <= ~out_clk2;
                    out_clk2_cnt <= 0;
                    end
                end
                // 10kHz clock
                always@(posedge clk)
                begin
                    out_clk3_cnt <= out_clk3_cnt + 1;
			\end{lstlisting}
			\begin{lstlisting}[language=Verilog]
                    if (out_clk3_cnt == div_10khz) 
                    begin
                    out_clk3 <= ~out_clk3;
                    out_clk3_cnt <= 0;
                    end
                end
                // 0.5Hz clock
                always@(posedge clk)
                begin
                    out_clk4_cnt <= out_clk4_cnt + 1;
                    if (out_clk4_cnt == div_05hz) 
                    begin
                    out_clk4 <= ~out_clk4;
                    out_clk4_cnt <= 0;
                    end
                end
                endmodule
            \end{lstlisting}
			将上面程序使用 Xilinx Vivado 软件进行仿真结果如下所示:
			\newline
			\begin{figure}[htbp]
				\centering
				\includegraphics[width=0.91\textwidth]{figure/divclk_sim.jpg}
				\caption{分频模块仿真(因系统原因,无法直接仿真 0.5Hz 时钟,后使用比例控制的方法验证,时钟正常)}\label{fig:divclk_sim}
			\end{figure}
			\newpage
			仿真结果与预想结果一致,可以实现分频功能。
		\subsection{累加和波形产生模块的编写}
			累加器的作用是为最后产生的波形提供地址变量,同时频率控制字也将会在此发挥作 用,最后累加的值在频率控制字为 0 时,累加 1,在频率控制字非 0 时,直接累加频率控制 字。因此从而实现累加功能,当累加器累加满时,就会产生溢出,即完成一个信号周期。频 率控制字的作用就是用来控制累加器加到满的速度。

			针对波形的产生,我们首先需要一个时钟来作为控制,同时前面累加单元累加的地址 也要作为输入来控制波形的频率,最后输出一个波形,由于波形的产生使用 Verilog HDL 语 言直接编写难度较大,因此这里直接使用 IP 核来用作波形生成器。IP 核采用 ROM,2 端口 输入,分别作为时钟输入和地址输入,输出端口用于输出波形的值。ROM 中的内容即为不 同波形所对应的值,这些值将存在一个 coe 文件中,便于读取。结合板卡 D/A 转换模块的 特点,生成的 coe 文件宽度为 12,深度为 4096,该模块中涉及累加、波形产生共计 5 个子模 块,其整体构架如下图所示:
			\newline
			\begin{figure}[htbp]
				\centering
				\includegraphics[width=0.95\textwidth]{figure/adder_rtl.jpg}
				\caption{累加器单元内部结构图}\label{fig:adder_rtl}
			\end{figure}

			从上图中我们可以看出,累加器和波形产生模块外部只需要 2 个输入,4 个输出,内部则需要我们自己编程调用 IP 核来实现连接,这也充分体现了 FPGA 设计中模块化设计的 思想与 Verilog HDL 语言侧重描述性的特点。

			根据上面的介绍和原理,我们根据上方图形来进行程序的编写。具体程序如下:
			\newpage
			\begin{lstlisting}[language=Verilog]
			module full_adder(
				input clk,
				input[11:0] freq_ctl,
				output[7:0] sine_wave,
				output[7:0] square_wave,
				output[7:0] triangle_wave,
				output[7:0] sawtooth_wave
				);

				reg[11:0] addr;

				initial
				begin
					addr <= 12'd0;
				end

				always@(posedge clk)
				begin
					addr <= addr + freq_ctl;
				end

				sine_rom sine(
					.a(addr),
					.clk(clk),
					.spo(sine_wave)
				);
				square_rom square(
					.a(addr),
					.clk(clk),
					.spo(square_wave)
				);
				triangle_rom triangle(
					.a(addr),
					.clk(clk),
			\end{lstlisting}
			\begin{lstlisting}[language=Verilog]
					.spo(triangle_wave)
				);
				sawtooth_rom sawtooth(
					.a(addr),
					.clk(clk),
					.spo(sawtooth_wave)
				);
			endmodule
			\end{lstlisting}

			将上面的 Verilog HDL 代码配合 coe 文件使用 Xilinx Vivado 软件进行仿真,得到如下所示波形图:
			\newline
			\begin{figure}[htbp]
				\centering
				\includegraphics[width=0.95\textwidth]{figure/adder_sim.jpg}
				\caption{仿真后输出的波形(方波在仿真时输出三角,属于正常现象)}\label{fig:adder_sim}
			\end{figure}

			从上方波形图可以看出,模块工作正常(锯齿波因 coe 文件原因仿真时会出现三角波),
			可以产生我们需要的波形。
		\subsection{波形输出模块的编写}
			从第二部分的分析中,显然 4 种波形不能同时输出到 D/A 模块中,因此我们需要编写
			一个选择模块来对前面累加和波形输出模块输出的 4 种波形进行输出,选择模块的特征与数字逻辑电路中的数据选择器十分相似,但是与之不同的是,这里选择的有效状态将仅有 4 个。根据板卡的硬件配置,这里我们采用 4 个开关来进行控制,设开关打开时值为 1,关闭 时值为 0,控制的具体情况如下表:
			{\wuhao\songti
				\begin{table}[htb]
					\begin{center}
						\topcaption{开关值与输出波形的对应关系}
						\begin{tabular}{m{6cm} m{5cm}}
							\bottomrule
							一组开关值(两位) & 对应波形\\
							\hline
							00 & 正弦波 \\
							01 & 方波 \\
							10 & 三角波 \\
							11 & 锯齿波 \\
							\bottomrule
						\end{tabular}
					\end{center}
			\end{table}}
			从上表中可以看出,一种开关状态将会对应一个波形输出状态,如果我们把波形看作数据输入,开关的值看作查找值,这一设计就类似于一个查找表(LUT),而 FPGA 是十分 擅长此类运算的,因此这一设计将会十分好地发挥 FPGA 的性能。

			根据上面的设计方案编写程序,具体程序如下所示:
			\begin{lstlisting}[language=Verilog]
			module wave_out(
				input clk, //10khz clock
				input[1:0] wave_selector,
				input[11:0] freq_ctl,
				output reg[7:0] output_wave
				);
			
				wire[7:0] sin_input;
				wire[7:0] squ_input;
				wire[7:0] tri_input;
				wire[7:0] saw_input;
			
				full_adder full_adder_embed(
					.clk(clk),
					.freq_ctl(freq_ctl),
					.sine_wave(sin_input),
					.square_wave(squ_input),
					.triangle_wave(tri_input),
					.sawtooth_wave(saw_input)
					);

			\end{lstlisting}
			\begin{lstlisting}[language=Verilog]
				always@(posedge clk)
				begin
					case(wave_selector)
						2'd0: output_wave <= sin_input;
						2'd1: output_wave <= squ_input;
						2'd2: output_wave <= tri_input;
						2'd3: output_wave <= saw_input;
					endcase
				end
			endmodule
			\end{lstlisting}

			仿真程序如下所示:
			\begin{lstlisting}[language=Verilog]
			module wave_out_sim;
			reg[11:0] ctl;
			reg[1:0] select;
			wire[7:0] out;
			
			initial
			begin
				select = 2'b11;
				ctl = 12'd4;
			end
			// clockgen 10khz
				reg clk = 0;
				parameter period_clk = 10000;
				always
				begin
					#(period_clk/2) clk <= 1'b0;
					#(period_clk/2) clk <= 1'b1;
				end
			wave_out wave_out_tb(
				.clk(clk),
				.wave_selector(select),
			\end{lstlisting}
			\begin{lstlisting}[language=Verilog]
				.freq_ctl(ctl),
				.output_wave(out)
				);
			endmodule
			\end{lstlisting}

			这里详细说明一下仿真模块程序的编写,由于 IP 核在仿真时相对耗费时间,且仿真 时想要看到波形只能将前面的累加器模块调用到波形输出模块中,因此这里我没 有采用直接仿真的方法,而是把前面的波形用特殊值代替,后通过 keys 来观察最后的输出值,这样的仿真方法不破坏源程序,而且速度快且便于检查问题,并且可 以直达本质来验证该模块的选择功能。

			利用 Xilinx Vivado 软件与前面得累加器和波形输出模块配合仿真,可以得到如下图所示的仿真波形:
			\newline
			\begin{figure}[htbp]
				\centering
				\includegraphics[width=1\textwidth]{figure/wave_out_sim.jpg}
				\caption{选择模块仿真输出波形}\label{fig:wave_out_sim}
			\end{figure}

			结合仿真波形图,我们可以看到,开关可以有效地控制波形的输出,且前面产生的 4种波形可以正常输出,程序有效。
		\subsection{理论频率值计算模块的编写}
			输出信号频率的理论值$f_{the}$仅和频率控制字$K$和时钟频率大小$f$,以及 ROM 的深度$D$有关系,三者关系可以使用如下表达式进行描述:

			\begin{equation}
				f_{the}=\frac{K}{D}\cdot f
			\end{equation}
			因此理论值计算模块的计算功能较为简单,只需要 2 个输入和一个表达式即可确定(ROM 深度为固定值),为防止后续模块
			工作量变大,我在这一模块中加入了分位输出功能, 即把理论值按照个、十、百、千位进行输出,计算各位的数值时将采
			用求余算法分别获取各 个数位上的数值,这一算法也是编程时分离各个数位的常用方法。所以理论值计算模块有 2 个
			输入,4 个输出,其组成内部结构图如下所示:
			\newline
			\begin{figure}[htbp]
				\centering
				\includegraphics[width=0.95\textwidth]{figure/the_freq_rtl.jpg}
				\caption{理论频率值计算模块内部结构}\label{fig:the_freq_rtl}
			\end{figure}

			根据上面的设计方案编写程序,具体程序如下所示:
			\begin{lstlisting}[language=Verilog]
			module theory_freq(
				input clk,
				input[11:0] freq_ctl,
				output reg[3:0] thou_count,
				output reg[3:0] hund_count,
				output reg[3:0] ten_count,
				output reg[3:0] one_count
				);
				reg[26:0] theo_freq;

				always@(posedge clk)
				begin
					theo_freq <= freq_ctl * 10000 / 4096;
					one_count <= theo_freq % 10;
			\end{lstlisting}
			\begin{lstlisting}[language=Verilog]
					ten_count <= theo_freq / 10 % 10;
					hund_count <= theo_freq / 100 % 10;
					thou_count <= theo_freq / 1000;
				end
			endmodule
			\end{lstlisting}

			利用 Xilinx Vivado 软件人为操控频率控制字和时钟进行仿真,可以得到如下图所示的仿真波形:
			\newline
			\begin{figure}[htbp]
				\centering
				\includegraphics[width=0.95\textwidth]{figure/the_freq_sim.jpg}
				\caption{理论频率测量模块的仿真波形}\label{fig:the_freq_sim}
			\end{figure}

			从上面的仿真图形可以看出,理论测量值准确,因此推断该模块功能运行正常,后续可以使用。
		\subsection{测频模块程序的书写}
			测频就是测出产生信号的频率,但是我们的信号最后的输出值为 0-255 或 0-127,甚 至在后续还会有其他数值,因此输出为模拟信号值。所以想要测量信号的频率,我们必须要 从模拟信号中找到关键突变点,根据突变点来制定方案。在本次实验中,我先后采用了 3 种突变点来测量信号的频率,具体如下:
			\begin{enumerate}
				\item 利用 1 秒内信号的最高点出现的次数来测量频率。
				\item 利用中间数值将信号转换为时钟脉冲信号,通过 1 秒钟内整形后的信号通过的 数量来测量时钟的频率,原理类似于滞回比较器。
				\item 利用 1 秒内输出信号最高位的变化来测量频率。
			\end{enumerate}
			在实验中我综合比较了上述三种方案的优劣。第一种方案利用最高点,结合 coe 文件, 最高点可能是 255 也可能是 127,但是结果经过仿真时发现,每次实验最高点的数值不一定 相同,后将最高点范围小幅度扩大,情况也未见改善,所以第一种方案在实际应用中不可行。 第二种方案在第一种方案的基础上作了改进,首先寻找中间点对脉冲进行整形,测量整形后 脉冲的周期,然而程序中 coe 文件的极值不统一,因此整形的时候要针对不同的情况使用不 同的时钟,在综合时发现,该方案报了错误 ambigious clock,后咨询老师与专业工程师,此 类错误原因就是时钟使用过多,且内部模块结构相对复杂,因此第二种方案不采用。第三种 方案则是在第二种方案的基础上再次做出改进,直接利用最高位变化的次数进行测量,相比 于第二种方案的具体数值,最高位相对具有统一性,而且对于不同位数的最高位,可将其集 中在同一个时钟当中,经过仿真,该方案可行性较好。

			针对方案三,确定了选择最高位作为关键突变点,下面需要根据这一突变点来确定测 频的逻辑。我们的思路就是要测量 1 秒钟内该突变点突变的次数,所以首要问题,就是测量 的触发时钟选择突变点作为时钟还是 0.5Hz 作为触发时钟。若选择 0.5Hz 时钟作为触发时钟, 我们在书写时,将根据时钟上升沿来判断,但是上升沿只会在一瞬间到达,而在这一瞬间不 一定会有脉冲,所以不能使用 0.5Hz 作为触发时钟。选择突变点作为触发时钟,如果发生突 变的时候,0.5Hz 时钟的值为 1,则计数一次,否则不计数,这一方案则可以完美契合我们 的预想功能。因此我们书写程序时采用上述方案。和前面的理论模块一样,这里同样在这一模块中采用了求余方法获取实际测频的各个数位。

			根据上面的设计方案和时序图编写程序,具体程序如下所示:
			\begin{lstlisting}[language=Verilog]
			module real_freq(
				input clk,
				input[7:0] wave,
				output reg[3:0] thou_count,
				output reg[3:0] hund_count,
				output reg[3:0] ten_count,
				output reg[3:0] one_count
				);
				reg[19:0] freq_tmp;
			\end{lstlisting}
			\begin{lstlisting}[language=Verilog]
				reg[19:0] freq_count;
				reg half;
				initial
				begin
					freq_count <= 20'd0;
					freq_tmp <= 20'd0;
				end
			
				always@(*)
				begin
					if(wave <= 8'd127)
					begin
						half <= 1'b0;
					end
					else
					begin
						half <= 1'b1;
					end
				end
			
				always@(posedge half)
				begin
					if(clk == 0)
					begin
						freq_tmp <= freq_tmp + 1;
					end
					else
					begin
						if(freq_tmp != 0)
						begin
							freq_count <= freq_tmp;
						end
						freq_tmp <= 0;
					end
			\end{lstlisting}
			\begin{lstlisting}[language=Verilog]
				end
			
				always@(negedge clk)
				begin
					one_count <= freq_count % 10;
					ten_count <= (freq_count / 10) % 10;
					hund_count <= (freq_count / 100) % 10;
					thou_count <= freq_count / 1000;
				end
			
			endmodule
			\end{lstlisting}

			仿真程序如下所示:

			\begin{lstlisting}[language=Verilog]
			module real_freq_sim;
			reg[11:0] ctl;
			wire[7:0] sine;
			wire clk_10khz;
			wire clk_05hz;
			wire[3:0] thou;
			wire[3:0] hund;
			wire[3:0] ten;
			wire[3:0] one;
			// clockgen 100mhz
			reg clk = 0;
			parameter period_clk = 10;
			always begin
				#(period_clk/2) clk <= 1'b0;
				#(period_clk/2) clk <= 1'b1;
			end
			initial begin
				ctl = 12'd100;
			end
			\end{lstlisting}
			\begin{lstlisting}[language=Verilog]
			divclk divclk_tb(
				.clk(clk), 
				.out_clk3(clk_10khz),
				.out_clk4(clk_05hz)
				);
			full_adder full_adder_tb(
				.clk(clk_10khz),
				.freq_ctl(ctl),
				.sine_wave(sine)
				);
			real_freq real_freq_tb(
				.clk(clk_05hz),
				.wave(sine),
				.thou_count(thou),
				.hund_count(hund),
				.ten_count(ten),
				.one_count(one)
				);
			endmodule
			\end{lstlisting}

			利用 Xilinx Vivado 人为操控时钟和0.5Hz时钟进行仿真,可得到下图所示的结果:
			\newline
			\begin{figure}[htbp]
				\centering
				\includegraphics[width=0.72\textwidth]{figure/real_freq_sim.jpg}
				\caption{测频模块仿真结果}\label{fig:real_freq_sim}
			\end{figure}

		\subsection{显示模块的程序编写}
			显示模块的功能就是把各个位显示到对应的数码管上,其中理论值显示在左侧 4 位,实际值显示在右侧 4 位。

			由于显示模块的管脚分配十分特殊,4 个数码管为一组,分 2 组控制,因此如果采用分位静态显示的方法会发生引脚配置不够的问题,因此这里采用动态扫描的方法。由于 4 个数码管用一组管脚控制,因此无法保证四个数码管一起显示。但是模块设置的输入为各个 位数据一起输入,因此这里我采用视觉暂留原理,以系统时钟作为模块的启动时钟,这样每 个数位亮起来的时候间隔仅为 10-8 秒,基本可以忽略,人眼认为是同时亮起,同时这一设计 也实现了并行输入到模块中的数据串行输出。因此这里采用上述动态扫描方法。

			根据上面方案使用 Verilog HDL 语言编写出程序如下:
			\begin{lstlisting}[language=Verilog]
			module seg_ctl(
				input clk, // 1khz clock
				input[31:0] disp_data_pool, // 8*4
				output reg[7:0] seg0, // left
				output reg[7:0] seg1, // right
				output reg[7:0] seg_flag // seg choose
				);
				reg[3:0] disp_data; // the data needed to display
				reg[2:0] dyn_disp_flag; // dynamic refresh
				// initialization
				initial
				begin
					seg0 = 0;
					seg1 = 0;
					seg_flag = 8'b00000001; // which seg to display
					disp_data = 0;
					dyn_disp_flag = 0;
				end
				// dynamic refresh
				always@(posedge clk)
				begin
					dyn_disp_flag <= dyn_disp_flag + 1;
					case (dyn_disp_flag)
			\end{lstlisting}
			\begin{lstlisting}[language=Verilog]
						3'b000:
						begin
						disp_data <= disp_data_pool[3:0];
						seg_flag <= 8'b00000001;
						end
						3'b001:
						begin
						disp_data <= disp_data_pool[7:4];
						seg_flag <= 8'b00000010;
						end
						3'b010:
						begin
						disp_data <= disp_data_pool[11:8];
						seg_flag <= 8'b00000100;
						end
						3'b011:
						begin
						disp_data <= disp_data_pool[15:12];
						seg_flag <= 8'b00001000;
						end
						3'b100:
						begin
						disp_data <= disp_data_pool[19:16];
						seg_flag <= 8'b00010000;
						end
						3'b101:
						begin
						disp_data <= disp_data_pool[23:20];
						seg_flag <= 8'b00100000;
						end
						3'b110:
						begin
						disp_data <= disp_data_pool[27:24];
						seg_flag <= 8'b01000000;
						end
			\end{lstlisting}
			\begin{lstlisting}[language=Verilog]
						3'b111:
						begin
						disp_data <= disp_data_pool[31:28];
						seg_flag <= 8'b10000000;
						end
						default:
						begin
						disp_data <= 0;
						seg_flag <= 8'b00000000;
						end
					endcase
				end
			
				always@(seg_flag)
				begin
					if (seg_flag > 8'b00001000)
					begin
					case (disp_data)
						4'h0 : seg0 <= 8'hfc;
						4'h1 : seg0 <= 8'h60;
						4'h2 : seg0 <= 8'hda;
						4'h3 : seg0 <= 8'hf2;
						4'h4 : seg0 <= 8'h66;
						4'h5 : seg0 <= 8'hb6;
						4'h6 : seg0 <= 8'hbe;
						4'h7 : seg0 <= 8'he0;
						4'h8 : seg0 <= 8'hfe;
						4'h9 : seg0 <= 8'hf6;
						4'ha : seg0 <= 8'hee;
						4'hb : seg0 <= 8'h3e;
						4'hc : seg0 <= 8'h9c;
						4'hd : seg0 <= 8'h7a;
						4'he : seg0 <= 8'h9e;
						4'hf : seg0 <= 8'h8e;
					endcase
			\end{lstlisting}
			\begin{lstlisting}[language=Verilog]
					end
					else
					begin
					case (disp_data)
						4'h0 : seg1 <= 8'hfc;
						4'h1 : seg1 <= 8'h60;
						4'h2 : seg1 <= 8'hda;
						4'h3 : seg1 <= 8'hf2;
						4'h4 : seg1 <= 8'h66;
						4'h5 : seg1 <= 8'hb6;
						4'h6 : seg1 <= 8'hbe;
						4'h7 : seg1 <= 8'he0;
						4'h8 : seg1 <= 8'hfe;
						4'h9 : seg1 <= 8'hf6;
						4'ha : seg1 <= 8'hee;
						4'hb : seg1 <= 8'h3e;
						4'hc : seg1 <= 8'h9c;
						4'hd : seg1 <= 8'h7a;
						4'he : seg1 <= 8'h9e;
						4'hf : seg1 <= 8'h8e;
					endcase
					end
				end
			endmodule
			\end{lstlisting}
			因为程序在仿真中比较难进行结果确认,所以直接烧录测试,测试后模块功能运行正常,可以正常使用。
		\subsection{按键输入模块编写}
			按键消抖通常的按键所用开关为机械弹性开关,当机械触点断开、闭合时,由于机械触点的弹性作用,一个按键开关在闭合时不会马上稳定地接通,在断开时也不会一下子断开。因而在闭合及断开的瞬间均伴随有一连串的抖动,为了不产生这种现象而作的措施就是按键消抖。

			在本模块中,利用了一个深度为3的移位寄存器,该移位寄存器由一个中速时钟($T=20ms$)驱动,结合现在和过去的两个状态来判断按键是否被按下,同时带有一个边缘检测,只会在按键按下的瞬间发送有效信号,按键一直按下不会保持输出有效信号。

			根据上面原理使用 Verilog HDL 语言编写出程序如下:
			\begin{lstlisting}[language=Verilog]
			module key_read(
				input clk,
				input[5:0] btn_input,//5 buttons + 1 reset
				output[5:0] btn_output
				);

				// button initialization
				reg[5:0] btn_status_0 = 0;
				reg[5:0] btn_status_1 = 0;
				reg[5:0] btn_status_2 = 0;
				// btn_out
				assign btn_output = (btn_status_2 & btn_status_1 & ~btn_status_0) | (btn_status_2 & btn_status_1 & btn_status_0) | (~btn_status_2 & btn_status_1 & btn_status_0);
				
				// status passer
				always @(posedge clk)
				begin
					btn_status_0 <= btn_input;
					btn_status_1 <= btn_status_0;
					btn_status_2 <= btn_status_1;
				end
			endmodule
			\end{lstlisting}

			因为程序在仿真中比较难进行结果确认,所以直接烧录测试,测试后模块功能运行正常,可以正常使用。
	\section{正弦波 AM 调制部分的编写}
		\subsection{分频模块需求分析}
			AM 调制和前面的 DDS 一样,考虑到示波器条件限制,因此我们将会采用分频模块,程序与前面的分频部分相同,因此这里不再赘 述。
		\subsection{AM调制模块编写}
			从 AM 调制的原理可以看出,AM 调制需要产生频率较大的载波信号和频率较小的传 输信号,参考 DDS 设计部分,这里采用与 DDS 产生正弦波形相同的原理进行编写,即使用 IP 核、对应的 coe 文件和频率控制字来生成对应的波形。通过上面
			的方案,我们可以产生 2 种频率的正弦波。

			生成正弦波的代码如下所示:
			\begin{lstlisting}[language=Verilog]
			module am_gen(
				input clk,
				output signed[7:0] sin_s,
				output signed[7:0] sin_c
				);

				parameter freq_s = 7'd64;
				parameter freq_c = 11'd1920;
				parameter cnt_width = 8'd15;
				
				reg [cnt_width-1:0] cnt_s;
				reg [cnt_width-1:0] cnt_c;

				wire [11:0] addr_s;
				wire [11:0] addr_c;

				wire signed[7:0] raw_s;
				wire signed[7:0] raw_c;

				initial
				begin
					cnt_s = 0;
					cnt_c = 0;
				end

				always @(posedge clk) 
				begin
			\end{lstlisting}
			\begin{lstlisting}[language=Verilog]
					cnt_s <= cnt_s + freq_s;
					cnt_c <= cnt_c + freq_c; 
				end

				assign addr_s = cnt_s[cnt_width-1:cnt_width-12]; 
				assign addr_c = cnt_c[cnt_width-1:cnt_width-12]; 

				assign sin_s = raw_s + 8'd128;
				assign sin_c = raw_c + 8'd128;

				sine_rom sine_am_s(
					.a(addr_s),
					.clk(clk),
					.spo(raw_s)
				);

				sine_rom sine_am_c(
					.a(addr_c),
					.clk(clk),
					.spo(raw_c)
				);

			endmodule
			\end{lstlisting}

			仿真程序如下所示:

			\begin{lstlisting}[language=Verilog]
			module am_gen_sim;

			wire[7:0] ss;
			wire[7:0] sc;
			
			// clockgen 10khz
			reg clk = 0;
			parameter period_clk = 10000;
			\end{lstlisting}
			\begin{lstlisting}[language=Verilog]
			always
			begin
				#(period_clk/2) clk <= 1'b0;
				#(period_clk/2) clk <= 1'b1;
			end
			
			am_gen am_gen_tb(
				.clk(clk),
				.sin_s(ss),
				.sin_c(sc)
			);
			endmodule
			\end{lstlisting}

			利用 Xilinx Vivado 人为输入时钟进行仿真,可得到下图所示的结果:
			\newline
			\begin{figure}[htbp]
				\centering
				\includegraphics[width=0.95\textwidth]{figure/am_gen_sim.jpg}
				\caption{AM调制正弦波生成仿真结果}\label{fig:am_gen_sim}
			\end{figure}

			然后我们需要将 2 个正弦波进行相乘运算,这里我们使用系统自带的乘法器 IP 核来
			完成运算。IP 核调用方式与产生波形的 ROM 的调用方式相同,其具体参数如下所示:
			\newpage
			\begin{figure}[htbp]
				\centering
				\includegraphics[width=0.75\textwidth]{figure/mult_setting.jpg}
				\caption{A乘法器 IP 核配置}\label{fig:mult_setting}
			\end{figure}

			生成乘法器 IP 核后,相乘可以得到一个位宽为 16 的数,后经过转换可以输出模拟的正弦波形。

			然而,由于我们的 DAC 模块只能输出 8 位无符号数,因此为保证波形能够在示波器上面正常输出,我们将最后得到的 16 位数据取高 8 位输入到 DAC 模块中,低八位数据舍 掉作为误差。但是前面的高八位仍旧为有符号数,所以需要做一个有符号数转无符号数的转 换。我们知道,8 位有符号数的最小值为-128,最大值为+127,因此转换时我们只需要加 128 即可将其范围转换为 0-255,刚好为无符号 8 位数的范围。

			该方案的构思电路图如下所示:
			\newline
			\begin{figure}[htbp]
				\centering
				\includegraphics[width=1\textwidth]{figure/am_top_stru.jpg}
				\caption{AM 调制部分的电路构思图设计}\label{fig:am_top_stru}
			\end{figure}

			根据上面的设计思想和构思图,我们可以编写出可以在世博器上面输出调制波形的顶 层模块的程序如下所示:
			\newpage
			\begin{lstlisting}[language=Verilog]
			module am_out(
				input clk,
				output reg[7:0] am_wave
				);
				wire[7:0] sc;
				wire[7:0] ss;
				wire[15:0] raw;
				reg[7:0] raw_tmp;
				am_gen am_gen_out(
					.clk(clk),
					.sin_s(ss),
					.sin_c(sc)
					);
				mult mult_am(
					.CLK(clk),
					.A(ss),
					.B(sc),
					.P(raw)
					);
				always@(posedge clk)
				begin
					raw_tmp = raw[15:8];
					am_wave = raw_tmp + 8'd128;
				end
			endmodule
			\end{lstlisting}

			其中时钟的大小为分频后的 10kHz。

			仿真程序如下所示:
			\begin{lstlisting}[language=Verilog]
			module am_out_sim;

			wire[7:0] out;
			// clockgen 10khz
			\end{lstlisting}
			\begin{lstlisting}[language=Verilog]
			reg clk = 0;
			parameter period_clk = 10000;
			always
			begin
				#(period_clk/2) clk <= 1'b0;
				#(period_clk/2) clk <= 1'b1;
			end
			
			am_out am_out_tb(
				.clk(clk),
				.am_wave(out)
			);
			endmodule
			\end{lstlisting}

			对上面的模块使用 Xilinx Vivado 自带仿真平台进行仿真,可以得到如下所示的波形:
			\newline
			\begin{figure}[htbp]
				\centering
				\includegraphics[width=0.95\textwidth]{figure/am_out_sim.jpg}
				\caption{AM 调制仿真波形结果}\label{fig:am_out_sim}
			\end{figure}

			经过检验,上面波形满足程序预想的逻辑,而且输出的调制波形形状较为理想,后面将会把上面的程序烧入开发板进行验证。

	\section{UART模块编写}
		\subsection{UART数据处理的思路}

		首先约定波特率,clk频率除以波特率为一帧所包含的时钟个数。其次是抓取起始位的边缘并对接收数据进行计数,之后由逻辑函数串行转并行,接收信号使用延迟两个时钟的d1,更稳定,防止亚稳态。
		\subsection{UART接收模块}

		原理在上文已经详细解释,这里不再赘述。
		\begin{lstlisting}[language=Verilog]
		module uart_rx(
			input clk, // 100mhz clock
			input uart_rxd, // raw data
			output reg[7:0] uart_data,
			output reg uart_rx_done
			);
		
			reg uart_rxd_tmp0;
			reg uart_rxd_tmp1;
			reg rx_flag;
			reg[3:0] rx_cnt;
			reg[15:0] clk_cnt;
			reg[7:0] rx_data;
			reg[3:0] last_rx_cnt;
			wire start;
			// config
			parameter CLK_FREQ = 100_000_000; 
			parameter UATR_FREQ = 9600; 
			parameter BSP_CNT = CLK_FREQ/UATR_FREQ;
		
			initial 
			begin
				rx_cnt = 0;
				rx_flag = 0;
				uart_rxd_tmp0 = 0;
				uart_rxd_tmp1 = 0;
		\end{lstlisting}
		\begin{lstlisting}[language=Verilog]
				clk_cnt = 0;
				rx_data = 0;
				last_rx_cnt = 0;
				uart_rx_done = 0;
			end
		
			// start
			assign start = uart_rxd_tmp1 & (~uart_rxd_tmp0);
			always @(posedge clk)
			begin 
				uart_rxd_tmp0 <= uart_rxd; 
				uart_rxd_tmp1 <= uart_rxd_tmp0; 
			end
		
			//rx_flag 
			always @(posedge clk)
			begin 
				if(start)
				begin
					rx_flag <= 1'd1;
				end
				else if(rx_cnt == 4'd9 && clk_cnt == BSP_CNT/2 - 1'd1)
				begin
					rx_flag <= 1'd0; 
				end
			end
		
			// clk_cnt 
			always @(posedge clk)
			begin 
				if (rx_flag)
				begin 
					if (clk_cnt < BSP_CNT - 1'd1)
					begin
						clk_cnt <= clk_cnt + 1'd1;
		\end{lstlisting}
		\begin{lstlisting}[language=Verilog]
					end
					else
					begin
						clk_cnt <= 16'd0; 
					end
				end 
				else
				begin
					clk_cnt <= 16'd0;
				end
			end
			
			// rx_cnt
			always @(posedge clk)
			begin 
				last_rx_cnt <= rx_cnt;
				if (rx_flag)
				begin 
					if (clk_cnt == BSP_CNT - 1'd1)
					begin
						rx_cnt <= rx_cnt + 4'd1;
					end
				end 
				else
				begin
					rx_cnt <= 4'd0;
				end 
			end
		
			// rx_data
			always @(posedge clk)
			begin 
				if (rx_flag)
				begin
					if (clk_cnt == BSP_CNT/2)
		\end{lstlisting}
		\begin{lstlisting}[language=Verilog]
					begin 
						case(rx_cnt)
							4'd1: rx_data[0] <= uart_rxd_tmp1; 
							4'd2: rx_data[1] <= uart_rxd_tmp1; 
							4'd3: rx_data[2] <= uart_rxd_tmp1; 
							4'd4: rx_data[3] <= uart_rxd_tmp1; 
							4'd5: rx_data[4] <= uart_rxd_tmp1; 
							4'd6: rx_data[5] <= uart_rxd_tmp1; 
							4'd7: rx_data[6] <= uart_rxd_tmp1; 
							4'd8: rx_data[7] <= uart_rxd_tmp1;
							default:; 
						endcase
					end
				end
				else
				begin 
					rx_data <= 8'd0;
				end
			end
		
			//rx_data 
			always @(posedge clk)
			begin 
				if (rx_cnt == 4'd9 && last_rx_cnt != 4'd9)
				begin 
					uart_rx_done <= 1'd1;
					uart_data <= rx_data;
				end
				else 
				begin
					uart_rx_done <= 1'd0;  
					uart_data <= 0;
				end
			end
		endmodule
		\end{lstlisting}

		接收控制模块编写:

		\begin{lstlisting}[language=Verilog]
		module recv_8byte(
			input clk,
			input bt_rxd,
			output reg[31:0] data_32bit,
			output refresh
			);
		
			wire bt_rx_done;
			wire[7:0] bt_recv_data;
			reg[3:0] recv_data_cache;
			reg test = 0;
		
			initial
			begin
				data_32bit = 32'd0;
				recv_data_cache = 0;
			end
		
			uart_rx uart_rx_bt(
				.clk(clk),
				.uart_rxd(bt_rxd),
				.uart_data(bt_recv_data),
				.uart_rx_done(bt_rx_done)
				);
		
			assign refresh = bt_rx_done;
		
			always@(posedge clk)
			begin
				if (bt_rx_done == 1)
				begin
					case(bt_recv_data)
						8'd48: recv_data_cache = 4'h0;
		\end{lstlisting}
		\begin{lstlisting}[language=Verilog]
						8'd49: recv_data_cache = 4'h1;
						8'd50: recv_data_cache = 4'h2;
						8'd51: recv_data_cache = 4'h3;
						8'd52: recv_data_cache = 4'h4;
						8'd53: recv_data_cache = 4'h5;
						8'd54: recv_data_cache = 4'h6;
						8'd55: recv_data_cache = 4'h7;
						8'd56: recv_data_cache = 4'h8;
						8'd57: recv_data_cache = 4'h9;
						default: recv_data_cache = 4'hf;
					endcase
		
					data_32bit[31:28] <= data_32bit[27:24];
					data_32bit[27:24] <= data_32bit[23:20];
					data_32bit[23:20] <= data_32bit[19:16];
					data_32bit[19:16] <= data_32bit[15:12];
					data_32bit[15:12] <= data_32bit[11:8];
					data_32bit[11:8] <= data_32bit[7:4];
					data_32bit[7:4] <= data_32bit[3:0];
					data_32bit[3:0] <= recv_data_cache;
				end
			end
		endmodule
		\end{lstlisting}

		\subsection{UART发送模块编写}

		原理在上文已经详细解释,这里不再赘述。
		\begin{lstlisting}[language=Verilog]
		module uart_tx(
			input clk,
			input[7:0] uart_data,
			input uart_tx_en,
			output reg uart_txd,
			output uart_tx_busy,
		\end{lstlisting}
		\begin{lstlisting}[language=Verilog]
			output uart_tx_active
			);
		
			reg uart_en_tmp0;
			reg uart_en_tmp1;
			reg tx_flag;
			reg[3:0] tx_cnt; // data being transmitted
			reg[15:0] clk_cnt;
			reg[7:0] tx_data; // a caching layer
			wire start; // start transmission flag
			parameter CLK_FREQ = 100_000_000; 
			parameter UART_FREQ = 9600; 
			parameter BSP_CNT = CLK_FREQ/UART_FREQ;
		
			// uart_tx_busy 
			assign uart_tx_busy = tx_flag; 
		
			// start
			assign start = uart_en_tmp0 & (~uart_en_tmp1);
			assign uart_tx_active = start;
			always@(posedge clk)
			begin
				uart_en_tmp0 <= uart_tx_en; 
				uart_en_tmp1 <= uart_en_tmp0; 
			end
		
			// clk_cnt 
			always @(posedge clk)
			begin 
				if(tx_flag)
				begin 
					if(clk_cnt < BSP_CNT - 1'd1)
					begin
						clk_cnt <= clk_cnt + 1'd1;
					end
		\end{lstlisting}
		\begin{lstlisting}[language=Verilog]
					else
					begin
						clk_cnt <= 16'd0;
					end 
				end 
				else 
					clk_cnt <= 16'd0; 
			end
		
			// tx_cnt 
			always @(posedge clk)
			begin 
				if(tx_flag)
				begin 
					if(clk_cnt == BSP_CNT - 1'd1)
					begin
						tx_cnt <= tx_cnt + 4'd1;
					end
				end 
				else
				begin
					tx_cnt <= 4'd0;
				end
			end
			// uart_txd 
			always @(posedge clk)
			begin 
				if(tx_flag)
				begin
				case(tx_cnt) 
					4'd0: uart_txd <= 1'd0; 
					4'd1: uart_txd <= tx_data[0]; 
					4'd2: uart_txd <= tx_data[1]; 
					4'd3: uart_txd <= tx_data[2]; 
					4'd4: uart_txd <= tx_data[3];
		\end{lstlisting}
		\begin{lstlisting}[language=Verilog]
					4'd5: uart_txd <= tx_data[4]; 
					4'd6: uart_txd <= tx_data[5]; 
					4'd7: uart_txd <= tx_data[6]; 
					4'd8: uart_txd <= tx_data[7]; 
					4'd9: uart_txd <= 1'd1; 
					default: uart_txd <= 1'd1; 
					endcase
				end 
				else 
					uart_txd <= 1'd1;
			end
		
			//tx_data tx_flag 
			always @(posedge clk)
			begin 
				if(start == 1'b1)
				begin 
					tx_data <= uart_data; 
					tx_flag <= 1'b1; 
				end 
				else if ((tx_cnt == 4'd9) && (clk_cnt == BSP_CNT -1'd1))
				begin 
					tx_flag <= 1'b0; 
					tx_data <= 8'd0; 
				end 
			end
		endmodule
		\end{lstlisting}

		接收控制模块编写:

		\begin{lstlisting}[language=Verilog]
		module send_8byte(
			input clk,
			input en,
			input[63:0] ascii_data,
		\end{lstlisting}
		\begin{lstlisting}[language=Verilog]
			output bt_txd,
			output reg done
			);
		
			reg bt_tx_en;
			wire bt_tx_busy;
			wire bt_tx_active;
			reg[3:0] send_count = 4'd0;
			reg[7:0] bt_send_data;
			reg clear;
			reg en_tmp0;
			reg en_tmp1;
			wire start;
			reg send_flag;
		
			uart_tx uart_tx_bt(
				.clk(clk),
				.uart_data(bt_send_data),
				.uart_tx_en(bt_tx_en),
				.uart_txd(bt_txd),
				.uart_tx_busy(bt_tx_busy),
				.uart_tx_active(bt_tx_active)
				);
		
			assign start = en_tmp0 & (~en_tmp1);
			always@(posedge clk)
			begin
				en_tmp0 <= en; 
				en_tmp1 <= en_tmp0; 
			end
		
			always@(posedge clk)
			begin
				if (start == 1'b1)
				begin
		\end{lstlisting}
		\begin{lstlisting}[language=Verilog]
					send_flag <= 1'b1;
				end
				else if (done == 1)
				begin
					send_flag <= 1'b0;
				end
			end
		
			always@(negedge bt_tx_busy or posedge done)
			begin
				if (send_count < 4'd8)
				begin
					send_count <= send_count + 4'd1;
				end
				if (done == 1)
				begin
					send_count <= 0;
				end
			end
		
			always@(posedge clk)
			begin
				if (send_flag == 1)
				begin
					if (bt_tx_busy == 0 && done == 0)
					begin
						case(send_count)
							4'd0: 
							begin
								bt_tx_en <= 1;
								bt_send_data <= ascii_data[63:56];
								done <= 0;
							end
							4'd1:
							begin
		\end{lstlisting}
		\begin{lstlisting}[language=Verilog]
								bt_tx_en <= 1;
								bt_send_data <= ascii_data[55:48];
								done <= 0;
							end
							4'd2:
							begin
								bt_tx_en <= 1;
								bt_send_data <= ascii_data[47:40];
								done <= 0;
							end
							4'd3: 
							begin
								bt_tx_en <= 1;
								bt_send_data <= ascii_data[39:32];
								done <= 0;
							end
							4'd4: 
							begin
								bt_tx_en <= 1;
								bt_send_data <= ascii_data[31:24];
								done <= 0;
							end
							4'd5: 
							begin
								bt_tx_en <= 1;
								bt_send_data <= ascii_data[23:16];
								done <= 0;
							end
							4'd6: 
							begin
								bt_tx_en <= 1;
								bt_send_data <= ascii_data[15:8];
								done <= 0;
							end
							4'd7: 
		\end{lstlisting}
		\begin{lstlisting}[language=Verilog]
							begin
								bt_tx_en <= 1;
								bt_send_data <= ascii_data[7:0];
								done <= 0;
							end
							4'd8:
							begin
								done <= 1;
								bt_tx_en <= 0;
								clear <= 1;
							end
						endcase
					end
					else
					begin
						bt_tx_en <= 0;
					end
				end
				else
				begin
					if (done != 0)
					begin
						done <= 0;
					end
				end
			end
		endmodule
		\end{lstlisting}
		\subsection{UART顶层模块编写}
		\begin{figure}[htbp]
			\centering
			\includegraphics[width=0.5\textwidth]{figure/uart_top.jpg}
			\caption{UART顶层模块结构}\label{fig:uart_top}
		\end{figure}
		\newpage
		控制代码如下,发送成功后会回传"Success!"字符串。RTL结构图如上:

		Verilog HDL代码如下:
		\begin{lstlisting}[language=Verilog]
		module bt_recv_response(
			input clk,
			input bt_rxd,
			output[31:0] bt_recv_data,
			output bt_txd
			);
		
			reg last_recv_data;
			reg[63:0] send_data;
			reg send_en;
			wire send_done;
			wire data_refresh;
		
			initial 
			begin
				last_recv_data = 0;
				send_en = 0;
				//Success!
				send_data[63:56] = 8'd83;
				send_data[55:48] = 8'd117;
				send_data[47:40] = 8'd99;
				send_data[39:32] = 8'd99;
				send_data[31:24] = 8'd101;
				send_data[23:16] = 8'd115;
				send_data[15:8] = 8'd115;
				send_data[7:0] = 8'd33;
			end
		
			recv_8byte bt_recv_8byte(
				.clk(clk),
				.bt_rxd(bt_rxd),
				.data_32bit(bt_recv_data),
		\end{lstlisting}
		\begin{lstlisting}[language=Verilog]
				.refresh(data_refresh)
				);
		
			send_8byte bt_send_8byte(
				.clk(clk)
				.en(send_en),
				.ascii_data(send_data),
				.bt_txd(bt_txd),
				.done(send_done)
				);
		
			always@(posedge clk)
			begin
				send_en = data_refresh;
			end
		endmodule
		\end{lstlisting}
	\section{顶层逻辑代码}
		\subsection{实现功能}
		\begin{enumerate}
			\item 利用右侧上下左右按钮实现数值/位切换
			\item 利用右侧中间按钮实现波形切换
			\item 利用reset按钮实现模式的切换
			\item 利用右侧的LED灯实现模型的显示
			\item 利用左侧拨码开关实现输入模式的切换
			\item 利用左侧的LED灯指示位数和波形
		\end{enumerate}
		\subsection{Verilog HDL代码}
		\begin{lstlisting}[language=Verilog]
		module top_dynamic(
			input clk,
			input[5:0] btn_in,
			input[7:0] switch,
			input burx,
			input surx,
		\end{lstlisting}
		\begin{lstlisting}[language=Verilog]
			output[7:0] seg0,
			output[7:0] seg1,
			output[7:0] seg_flag,
			output[7:0] LED0,
			output[7:0] LED1,
			output[7:0] toDAC,
			output DAC_ILE,
			output DAC_CS,
			output DAC_WR1,
			output DAC_WR2,
			output DAC_XFER,
			output butx,
			output sutx,
			output reg bt_master=1 , bt_sw_hw=0 , bt_rst=1 , bt_sw=1 , bt_pw=1
			);
		
			reg[5:0] btn_seg_flag;
			reg freq_ctl_flag;
			reg[31:0] num_data_pool;
		
			reg[1:0] adjust_ctl_flag;
		
			// DAC param initialization
			assign DAC_ILE = 1;
			assign DAC_CS = 0;
			assign DAC_WR1 = 0;
			assign DAC_WR2 = 0;
			assign DAC_XFER = 0;
		
			// clock divider
			wire clk_1ms;
			wire clk_20ms;
			wire clk_10khz;
			wire clk_05hz;
		
			divclk divclk_dds(
			\end{lstlisting}
			\begin{lstlisting}[language=Verilog]
				.clk(clk),
				.out_clk1(clk_1ms),
				.out_clk2(clk_20ms),
				.out_clk3(clk_10khz),
				.out_clk4(clk_05hz)
				);
			
			// button input
			wire[5:0] btn_out;
		
			key_read keyread_dds(
				.clk(clk_20ms),
				.btn_input(btn_in),
				.btn_output(btn_out)
				);
		
			// display
			wire[31:0] disp_data_pool;
			seg_ctl seg_ctl_dds(
				.clk(clk_1ms),
				.disp_data_pool(disp_data_pool),
				.seg0(seg0),
				.seg1(seg1),
				.seg_flag(seg_flag)
				);
		
			wire dac_mode;
			wire dds_clk;
			wire am_clk;
		
			assign dds_clk = clk_10khz & dac_mode;
			assign am_clk = clk_10khz & (~dac_mode);
		
			// dds
			wire[1:0] wave_select;
			wire[11:0] freq_ctl;
		\end{lstlisting}
		\begin{lstlisting}[language=Verilog]
			wire[7:0] dds_mux;
		
			wave_out wave_out_dds(
				.clk(dds_clk),
				.wave_selector(wave_select),
				.freq_ctl(freq_ctl),
				.output_wave(dds_mux)
				);
		
			// am
			wire[7:0] am_mux;
			am_out am_out_dynamic(
				.clk(am_clk),
				.am_wave(am_mux)
				);
		
			// dac mux
			dac_mux dac_mux_dynamic(
				.mode(dac_mode),
				.am(am_mux),
				.dds(dds_mux),
				.dac(toDAC)
				);
		
			// theoretical frequency & real frequency display data to display
			wire[31:0] freq_data_pool;
		
			freq_embed freq_embed_dds(
				.clk_10k(clk_10khz),
				.clk_05(clk_05hz),
				.wave_in(toDAC),
				.freq_ctl(freq_ctl),
				.freq_data(freq_data_pool)
				);
		
			// process freq_ctl_data that needed to display
		\end{lstlisting}
		\begin{lstlisting}[language=Verilog]
			wire[31:0] freq_ctl_data_pool;
		
			freq_ctl_data freq_ctl_data_dds(
				.clk(clk_10khz),
				.freq_ctl(freq_ctl),
				.freq_ctl_data(freq_ctl_data_pool)
				);
		
			// uart mux
			wire uart_rx;
			wire uart_tx;
			wire uart_mode;
		
			uart_mux uart_mux_dynamic(
				.mode(uart_mode),
				.rx(uart_rx),
				.tx(uart_tx),
				.bt_rx(burx),
				.bt_tx(butx),
				.serial_rx(surx),
				.serial_tx(sutx)
				);
			
			// uart
			wire[31:0] uart_recv_data;
			bt_recv_response bt_recv_response_dynamic(
				.clk(clk),
				.bt_rxd(uart_rx),
				.bt_recv_data(uart_recv_data),
				.bt_txd(uart_tx)
				);
		
			// main control function
			dynamic_ctl dynamic_ctl_dds(
				.clk_10khz(clk_10khz),
				.to_DAC(toDAC),
		\end{lstlisting}
		\begin{lstlisting}[language=Verilog]
				.btn_value(btn_out),
				.switch(switch),
				.freq_data_pool(freq_data_pool),
				.freq_ctl_data_pool(freq_ctl_data_pool),
				.uart_recv_data(uart_recv_data),
				.LED0(LED0),
				.LED1(LED1),
				.wave_select(wave_select),
				.dac_mode(dac_mode),
				.uart_mode(uart_mode),
				.freq_ctl(freq_ctl),
				.data(disp_data_pool)
				);
		endmodule
		\end{lstlisting}
\end{spacing}}