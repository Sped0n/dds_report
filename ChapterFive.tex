\songti\xiaosi{
\chapter{实验感想及总结}
\begin{spacing}{1.5}
\section{实验中遇到的问题以及解决方法}
\subsection{实验初期发现示波器无波形显示且仿真正常}
解决方法:

我先对于上面的 DDS 部分所有程序进行了 RTL 检验,发现 RTL 图连接正常, 且引脚配置和之前自己设计的相同,后检查综合器的 warning,在波形输出所有相关模块未 发现与之相关的可以警告。最后我使用 ILA 逻辑分析仪进行波形的抓取分析,在抓取时发 现,两个分频器没有信号输出,因此锁定问题在分频器处。后检查分频器程序,时序没有问 题,仿真正常。尝试把复位功能去掉之后再次抓取波形,发现此时分频器信号抓取正常,后 检查波形输出,输出正常,烧录板子验证,发现波形显示正常。经过分析原因,发现原来设 置的复位按键有抖动嫌疑,于是将reset按键更改为别的功能并增加按键防抖。

\subsection{实验代码出现时序冲突}
解决方法:

我一开始在两个always块里对同一个寄存器进行操作,在simulation环节并未报错,但在implementation环节报错,后来发现不可以跨always块对寄存器赋值,会造成时序冲突,重构代码后问题解决。

\section{实验收获与感受}
\subsection{实验硬件和软件使用感受}
软件方面。每家公司在生产自己的 FPGA 时都会有自己的 EDA 软件, 这次实验也让我体会到了 Xilinx 公司
的软件配置相对于国产平台还是强大一些。我们对比 诸多国产厂家的 EDA 软件,可以发现很多厂家的软件都
与 Xilinx 相似,但是软件的速度上 Xilinx vivado 软件更胜一筹,尤其是在仿真过程中,国产软件会
比 Xilinx 慢一些。 功能方面,Vivado 是我目前用过的最好的软件, 因为它的综合器综合后的代码相对符合我的设计。

硬件方面。本次实验中硬件配置良好,但是由于硬件中拨码开关在前面的同 学实验时用力过猛可能被按坏了,所以在我的实验中出现了复位功能失灵的情况。除此以外, 本次实验的硬件全部工作正常。同时我在实验中也十分重视对硬件的保护。

\subsection{实验收获与自身编程感受}
在本次实验中,我首先通过老师讲解和查阅相关文献,花了 2 天左右的时间来设计 DDS 的模块及连接方式,并且对于模块的数量、每个模块的功能都做到了心中有数。实验 过程中很多同学对我的做法都不是很理解,但是后面我对照程序图写程序而且仿真十分顺利 的时候,我成功地看到了我这一方法的优点。我做实验的时候,除 coe 文件出错和 ambigious clock 综合错误以外,剩余逻辑几乎都是一次仿真成功,为我节省了大量时间。因此,在实 验初期设计好一个模块设计图是十分重要的,而不是上手就写程序,不然写到后面就会发现 各个模块的连接会十分混乱。把前期工作做好了以后,后面写起来就会十分轻松。

本次实验中除 matlab 生成 coe 文件使用了网上参考的程序,AM 调制借鉴了网络参 考的架构外,剩余程序全部为自己所写,因此通过这次实验,我的逻辑分析、电路设计素质 以及编程能力都得到了很大的提高。同时,我也更加深刻地体会到了在集创杯培训上吴国盛 工程师说过的:Verilog HDL 语言是描述性语言,而不是编程性语言这句话的实质。我也更 加深入地理解了 Verilog HDL 中 wire 类型和 reg 类型的区别,以及阻塞赋值和非阻塞赋值 的区别,我在第一次接触 Verilog HDL 语言时,就经常在这里出现错误,从而造成仿真波形 时序混乱和错位。但是经过这次实验,我相信我学到了很多。

\subsection{实验总结}
通过本次的实验,我第一次近距离接触到了平常很难买到的高价 FPGA,并且成功使用 Vivado 软件完成了 DDS 的相关操作,学习了 DDS 相关理论,让我更深一层地理解了 Verilog HDL 语言的语法及相关性质,同时也暴露了我的 理论学习存在较大问题。当然,我未来的 梦想是嵌入式和 FPGA 工程师,仅靠这一次实验的锻炼远远不够,我后面还会更深一步了 解 FPGA 的相关架构以及电路设计的相关知识,为学校争光,为祖国未来 FPGA 事业的发 展贡献自己的一份力量。

\end{spacing}}